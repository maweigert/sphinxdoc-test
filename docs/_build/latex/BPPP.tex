% Generated by Sphinx.
\def\sphinxdocclass{report}
\documentclass[letterpaper,10pt,english]{sphinxmanual}
\usepackage[utf8]{inputenc}
\DeclareUnicodeCharacter{00A0}{\nobreakspace}
\usepackage{cmap}
\usepackage[T1]{fontenc}
\usepackage{amsfonts}
\usepackage{babel}
\usepackage{times}
\usepackage[Bjarne]{fncychap}
\usepackage{longtable}
\usepackage{sphinx}
\usepackage{multirow}
\usepackage{eqparbox}


\addto\captionsenglish{\renewcommand{\figurename}{Fig. }}
\addto\captionsenglish{\renewcommand{\tablename}{Table }}
\SetupFloatingEnvironment{literal-block}{name=Listing }



\title{BPPP Documentation}
\date{March 25, 2016}
\release{1.0}
\author{MW}
\newcommand{\sphinxlogo}{}
\renewcommand{\releasename}{Release}
\setcounter{tocdepth}{1}
\makeindex

\makeatletter
\def\PYG@reset{\let\PYG@it=\relax \let\PYG@bf=\relax%
    \let\PYG@ul=\relax \let\PYG@tc=\relax%
    \let\PYG@bc=\relax \let\PYG@ff=\relax}
\def\PYG@tok#1{\csname PYG@tok@#1\endcsname}
\def\PYG@toks#1+{\ifx\relax#1\empty\else%
    \PYG@tok{#1}\expandafter\PYG@toks\fi}
\def\PYG@do#1{\PYG@bc{\PYG@tc{\PYG@ul{%
    \PYG@it{\PYG@bf{\PYG@ff{#1}}}}}}}
\def\PYG#1#2{\PYG@reset\PYG@toks#1+\relax+\PYG@do{#2}}

\expandafter\def\csname PYG@tok@gd\endcsname{\def\PYG@tc##1{\textcolor[rgb]{0.63,0.00,0.00}{##1}}}
\expandafter\def\csname PYG@tok@gu\endcsname{\let\PYG@bf=\textbf\def\PYG@tc##1{\textcolor[rgb]{0.50,0.00,0.50}{##1}}}
\expandafter\def\csname PYG@tok@gt\endcsname{\def\PYG@tc##1{\textcolor[rgb]{0.00,0.27,0.87}{##1}}}
\expandafter\def\csname PYG@tok@gs\endcsname{\let\PYG@bf=\textbf}
\expandafter\def\csname PYG@tok@gr\endcsname{\def\PYG@tc##1{\textcolor[rgb]{1.00,0.00,0.00}{##1}}}
\expandafter\def\csname PYG@tok@cm\endcsname{\let\PYG@it=\textit\def\PYG@tc##1{\textcolor[rgb]{0.25,0.50,0.56}{##1}}}
\expandafter\def\csname PYG@tok@vg\endcsname{\def\PYG@tc##1{\textcolor[rgb]{0.73,0.38,0.84}{##1}}}
\expandafter\def\csname PYG@tok@vi\endcsname{\def\PYG@tc##1{\textcolor[rgb]{0.73,0.38,0.84}{##1}}}
\expandafter\def\csname PYG@tok@mh\endcsname{\def\PYG@tc##1{\textcolor[rgb]{0.13,0.50,0.31}{##1}}}
\expandafter\def\csname PYG@tok@cs\endcsname{\def\PYG@tc##1{\textcolor[rgb]{0.25,0.50,0.56}{##1}}\def\PYG@bc##1{\setlength{\fboxsep}{0pt}\colorbox[rgb]{1.00,0.94,0.94}{\strut ##1}}}
\expandafter\def\csname PYG@tok@ge\endcsname{\let\PYG@it=\textit}
\expandafter\def\csname PYG@tok@vc\endcsname{\def\PYG@tc##1{\textcolor[rgb]{0.73,0.38,0.84}{##1}}}
\expandafter\def\csname PYG@tok@il\endcsname{\def\PYG@tc##1{\textcolor[rgb]{0.13,0.50,0.31}{##1}}}
\expandafter\def\csname PYG@tok@go\endcsname{\def\PYG@tc##1{\textcolor[rgb]{0.20,0.20,0.20}{##1}}}
\expandafter\def\csname PYG@tok@cp\endcsname{\def\PYG@tc##1{\textcolor[rgb]{0.00,0.44,0.13}{##1}}}
\expandafter\def\csname PYG@tok@gi\endcsname{\def\PYG@tc##1{\textcolor[rgb]{0.00,0.63,0.00}{##1}}}
\expandafter\def\csname PYG@tok@gh\endcsname{\let\PYG@bf=\textbf\def\PYG@tc##1{\textcolor[rgb]{0.00,0.00,0.50}{##1}}}
\expandafter\def\csname PYG@tok@ni\endcsname{\let\PYG@bf=\textbf\def\PYG@tc##1{\textcolor[rgb]{0.84,0.33,0.22}{##1}}}
\expandafter\def\csname PYG@tok@nl\endcsname{\let\PYG@bf=\textbf\def\PYG@tc##1{\textcolor[rgb]{0.00,0.13,0.44}{##1}}}
\expandafter\def\csname PYG@tok@nn\endcsname{\let\PYG@bf=\textbf\def\PYG@tc##1{\textcolor[rgb]{0.05,0.52,0.71}{##1}}}
\expandafter\def\csname PYG@tok@no\endcsname{\def\PYG@tc##1{\textcolor[rgb]{0.38,0.68,0.84}{##1}}}
\expandafter\def\csname PYG@tok@na\endcsname{\def\PYG@tc##1{\textcolor[rgb]{0.25,0.44,0.63}{##1}}}
\expandafter\def\csname PYG@tok@nb\endcsname{\def\PYG@tc##1{\textcolor[rgb]{0.00,0.44,0.13}{##1}}}
\expandafter\def\csname PYG@tok@nc\endcsname{\let\PYG@bf=\textbf\def\PYG@tc##1{\textcolor[rgb]{0.05,0.52,0.71}{##1}}}
\expandafter\def\csname PYG@tok@nd\endcsname{\let\PYG@bf=\textbf\def\PYG@tc##1{\textcolor[rgb]{0.33,0.33,0.33}{##1}}}
\expandafter\def\csname PYG@tok@ne\endcsname{\def\PYG@tc##1{\textcolor[rgb]{0.00,0.44,0.13}{##1}}}
\expandafter\def\csname PYG@tok@nf\endcsname{\def\PYG@tc##1{\textcolor[rgb]{0.02,0.16,0.49}{##1}}}
\expandafter\def\csname PYG@tok@si\endcsname{\let\PYG@it=\textit\def\PYG@tc##1{\textcolor[rgb]{0.44,0.63,0.82}{##1}}}
\expandafter\def\csname PYG@tok@s2\endcsname{\def\PYG@tc##1{\textcolor[rgb]{0.25,0.44,0.63}{##1}}}
\expandafter\def\csname PYG@tok@nt\endcsname{\let\PYG@bf=\textbf\def\PYG@tc##1{\textcolor[rgb]{0.02,0.16,0.45}{##1}}}
\expandafter\def\csname PYG@tok@nv\endcsname{\def\PYG@tc##1{\textcolor[rgb]{0.73,0.38,0.84}{##1}}}
\expandafter\def\csname PYG@tok@s1\endcsname{\def\PYG@tc##1{\textcolor[rgb]{0.25,0.44,0.63}{##1}}}
\expandafter\def\csname PYG@tok@ch\endcsname{\let\PYG@it=\textit\def\PYG@tc##1{\textcolor[rgb]{0.25,0.50,0.56}{##1}}}
\expandafter\def\csname PYG@tok@m\endcsname{\def\PYG@tc##1{\textcolor[rgb]{0.13,0.50,0.31}{##1}}}
\expandafter\def\csname PYG@tok@gp\endcsname{\let\PYG@bf=\textbf\def\PYG@tc##1{\textcolor[rgb]{0.78,0.36,0.04}{##1}}}
\expandafter\def\csname PYG@tok@sh\endcsname{\def\PYG@tc##1{\textcolor[rgb]{0.25,0.44,0.63}{##1}}}
\expandafter\def\csname PYG@tok@ow\endcsname{\let\PYG@bf=\textbf\def\PYG@tc##1{\textcolor[rgb]{0.00,0.44,0.13}{##1}}}
\expandafter\def\csname PYG@tok@sx\endcsname{\def\PYG@tc##1{\textcolor[rgb]{0.78,0.36,0.04}{##1}}}
\expandafter\def\csname PYG@tok@bp\endcsname{\def\PYG@tc##1{\textcolor[rgb]{0.00,0.44,0.13}{##1}}}
\expandafter\def\csname PYG@tok@c1\endcsname{\let\PYG@it=\textit\def\PYG@tc##1{\textcolor[rgb]{0.25,0.50,0.56}{##1}}}
\expandafter\def\csname PYG@tok@o\endcsname{\def\PYG@tc##1{\textcolor[rgb]{0.40,0.40,0.40}{##1}}}
\expandafter\def\csname PYG@tok@kc\endcsname{\let\PYG@bf=\textbf\def\PYG@tc##1{\textcolor[rgb]{0.00,0.44,0.13}{##1}}}
\expandafter\def\csname PYG@tok@c\endcsname{\let\PYG@it=\textit\def\PYG@tc##1{\textcolor[rgb]{0.25,0.50,0.56}{##1}}}
\expandafter\def\csname PYG@tok@mf\endcsname{\def\PYG@tc##1{\textcolor[rgb]{0.13,0.50,0.31}{##1}}}
\expandafter\def\csname PYG@tok@err\endcsname{\def\PYG@bc##1{\setlength{\fboxsep}{0pt}\fcolorbox[rgb]{1.00,0.00,0.00}{1,1,1}{\strut ##1}}}
\expandafter\def\csname PYG@tok@mb\endcsname{\def\PYG@tc##1{\textcolor[rgb]{0.13,0.50,0.31}{##1}}}
\expandafter\def\csname PYG@tok@ss\endcsname{\def\PYG@tc##1{\textcolor[rgb]{0.32,0.47,0.09}{##1}}}
\expandafter\def\csname PYG@tok@sr\endcsname{\def\PYG@tc##1{\textcolor[rgb]{0.14,0.33,0.53}{##1}}}
\expandafter\def\csname PYG@tok@mo\endcsname{\def\PYG@tc##1{\textcolor[rgb]{0.13,0.50,0.31}{##1}}}
\expandafter\def\csname PYG@tok@kd\endcsname{\let\PYG@bf=\textbf\def\PYG@tc##1{\textcolor[rgb]{0.00,0.44,0.13}{##1}}}
\expandafter\def\csname PYG@tok@mi\endcsname{\def\PYG@tc##1{\textcolor[rgb]{0.13,0.50,0.31}{##1}}}
\expandafter\def\csname PYG@tok@kn\endcsname{\let\PYG@bf=\textbf\def\PYG@tc##1{\textcolor[rgb]{0.00,0.44,0.13}{##1}}}
\expandafter\def\csname PYG@tok@cpf\endcsname{\let\PYG@it=\textit\def\PYG@tc##1{\textcolor[rgb]{0.25,0.50,0.56}{##1}}}
\expandafter\def\csname PYG@tok@kr\endcsname{\let\PYG@bf=\textbf\def\PYG@tc##1{\textcolor[rgb]{0.00,0.44,0.13}{##1}}}
\expandafter\def\csname PYG@tok@s\endcsname{\def\PYG@tc##1{\textcolor[rgb]{0.25,0.44,0.63}{##1}}}
\expandafter\def\csname PYG@tok@kp\endcsname{\def\PYG@tc##1{\textcolor[rgb]{0.00,0.44,0.13}{##1}}}
\expandafter\def\csname PYG@tok@w\endcsname{\def\PYG@tc##1{\textcolor[rgb]{0.73,0.73,0.73}{##1}}}
\expandafter\def\csname PYG@tok@kt\endcsname{\def\PYG@tc##1{\textcolor[rgb]{0.56,0.13,0.00}{##1}}}
\expandafter\def\csname PYG@tok@sc\endcsname{\def\PYG@tc##1{\textcolor[rgb]{0.25,0.44,0.63}{##1}}}
\expandafter\def\csname PYG@tok@sb\endcsname{\def\PYG@tc##1{\textcolor[rgb]{0.25,0.44,0.63}{##1}}}
\expandafter\def\csname PYG@tok@k\endcsname{\let\PYG@bf=\textbf\def\PYG@tc##1{\textcolor[rgb]{0.00,0.44,0.13}{##1}}}
\expandafter\def\csname PYG@tok@se\endcsname{\let\PYG@bf=\textbf\def\PYG@tc##1{\textcolor[rgb]{0.25,0.44,0.63}{##1}}}
\expandafter\def\csname PYG@tok@sd\endcsname{\let\PYG@it=\textit\def\PYG@tc##1{\textcolor[rgb]{0.25,0.44,0.63}{##1}}}

\def\PYGZbs{\char`\\}
\def\PYGZus{\char`\_}
\def\PYGZob{\char`\{}
\def\PYGZcb{\char`\}}
\def\PYGZca{\char`\^}
\def\PYGZam{\char`\&}
\def\PYGZlt{\char`\<}
\def\PYGZgt{\char`\>}
\def\PYGZsh{\char`\#}
\def\PYGZpc{\char`\%}
\def\PYGZdl{\char`\$}
\def\PYGZhy{\char`\-}
\def\PYGZsq{\char`\'}
\def\PYGZdq{\char`\"}
\def\PYGZti{\char`\~}
% for compatibility with earlier versions
\def\PYGZat{@}
\def\PYGZlb{[}
\def\PYGZrb{]}
\makeatother

\renewcommand\PYGZsq{\textquotesingle}

\begin{document}

\maketitle
\tableofcontents
\phantomsection\label{index::doc}



\chapter{Introduction}
\label{index:introduction}\label{index:documentation-for-the-code}
This is something I want to say that is not in the docstring.


\section{Installing}
\label{installing::doc}\label{installing:installing}
Installing is easy via pip:

\begin{Verbatim}[commandchars=\\\{\}]
pip install bpm
\end{Verbatim}


\section{Basic Usage}
\label{usage::doc}\label{usage:basic-usage}
Here is how to use it:

\begin{Verbatim}[commandchars=\\\{\}]
\PYG{n}{m} \PYG{o}{=} \PYG{n}{myClass}\PYG{p}{(}\PYG{l+s+s2}{\PYGZdq{}}\PYG{l+s+s2}{hello}\PYG{l+s+s2}{\PYGZdq{}}\PYG{p}{)}
\end{Verbatim}
\index{meshgrid() (in module numpy)}

\begin{fulllineitems}
\phantomsection\label{usage:numpy.meshgrid}\pysiglinewithargsret{\bfcode{meshgrid}}{\emph{*xi}, \emph{**kwargs}}{}
Return coordinate matrices from coordinate vectors.

Make N-D coordinate arrays for vectorized evaluations of
N-D scalar/vector fields over N-D grids, given
one-dimensional coordinate arrays x1, x2,..., xn.

\DUspan{versionmodified}{Changed in version 1.9: }1-D and 0-D cases are allowed.
\begin{quote}\begin{description}
\item[{Parameters}] \leavevmode\begin{itemize}
\item {} 
\textbf{\texttt{x2,..., xn}} (\emph{\texttt{x1,}}) -- 1-D arrays representing the coordinates of a grid.

\item {} 
\textbf{\texttt{indexing}} (\emph{\texttt{\{'xy', 'ij'\}, optional}}) -- 
Cartesian (`xy', default) or matrix (`ij') indexing of output.
See Notes for more details.

\DUspan{versionmodified}{New in version 1.7.0.}


\item {} 
\textbf{\texttt{sparse}} (\emph{\texttt{bool, optional}}) -- 
If True a sparse grid is returned in order to conserve memory.
Default is False.

\DUspan{versionmodified}{New in version 1.7.0.}


\item {} 
\textbf{\texttt{copy}} (\emph{\texttt{bool, optional}}) -- 
If False, a view into the original arrays are returned in order to
conserve memory.  Default is True.  Please note that
\code{sparse=False, copy=False} will likely return non-contiguous
arrays.  Furthermore, more than one element of a broadcast array
may refer to a single memory location.  If you need to write to the
arrays, make copies first.

\DUspan{versionmodified}{New in version 1.7.0.}


\end{itemize}

\item[{Returns}] \leavevmode
\textbf{X1, X2,..., XN} --
For vectors \emph{x1}, \emph{x2},..., `xn' with lengths \code{Ni=len(xi)} ,
return \code{(N1, N2, N3,...Nn)} shaped arrays if indexing='ij'
or \code{(N2, N1, N3,...Nn)} shaped arrays if indexing='xy'
with the elements of \emph{xi} repeated to fill the matrix along
the first dimension for \emph{x1}, the second for \emph{x2} and so on.

\item[{Return type}] \leavevmode
ndarray

\end{description}\end{quote}
\paragraph{Notes}

This function supports both indexing conventions through the indexing
keyword argument.  Giving the string `ij' returns a meshgrid with
matrix indexing, while `xy' returns a meshgrid with Cartesian indexing.
In the 2-D case with inputs of length M and N, the outputs are of shape
(N, M) for `xy' indexing and (M, N) for `ij' indexing.  In the 3-D case
with inputs of length M, N and P, outputs are of shape (N, M, P) for
`xy' indexing and (M, N, P) for `ij' indexing.  The difference is
illustrated by the following code snippet:

\begin{Verbatim}[commandchars=\\\{\}]
xv, yv = meshgrid(x, y, sparse=False, indexing=\PYGZsq{}ij\PYGZsq{})
for i in range(nx):
    for j in range(ny):
        \PYGZsh{} treat xv[i,j], yv[i,j]

xv, yv = meshgrid(x, y, sparse=False, indexing=\PYGZsq{}xy\PYGZsq{})
for i in range(nx):
    for j in range(ny):
        \PYGZsh{} treat xv[j,i], yv[j,i]
\end{Verbatim}

In the 1-D and 0-D case, the indexing and sparse keywords have no effect.


\strong{See also:}

\begin{description}
\item[{\code{index\_tricks.mgrid()}}] \leavevmode
Construct a multi-dimensional ``meshgrid'' using indexing notation.

\item[{\code{index\_tricks.ogrid()}}] \leavevmode
Construct an open multi-dimensional ``meshgrid'' using indexing notation.

\end{description}


\paragraph{Examples}

\begin{Verbatim}[commandchars=\\\{\}]
\PYG{g+gp}{\PYGZgt{}\PYGZgt{}\PYGZgt{} }\PYG{n}{nx}\PYG{p}{,} \PYG{n}{ny} \PYG{o}{=} \PYG{p}{(}\PYG{l+m+mi}{3}\PYG{p}{,} \PYG{l+m+mi}{2}\PYG{p}{)}
\PYG{g+gp}{\PYGZgt{}\PYGZgt{}\PYGZgt{} }\PYG{n}{x} \PYG{o}{=} \PYG{n}{np}\PYG{o}{.}\PYG{n}{linspace}\PYG{p}{(}\PYG{l+m+mi}{0}\PYG{p}{,} \PYG{l+m+mi}{1}\PYG{p}{,} \PYG{n}{nx}\PYG{p}{)}
\PYG{g+gp}{\PYGZgt{}\PYGZgt{}\PYGZgt{} }\PYG{n}{y} \PYG{o}{=} \PYG{n}{np}\PYG{o}{.}\PYG{n}{linspace}\PYG{p}{(}\PYG{l+m+mi}{0}\PYG{p}{,} \PYG{l+m+mi}{1}\PYG{p}{,} \PYG{n}{ny}\PYG{p}{)}
\PYG{g+gp}{\PYGZgt{}\PYGZgt{}\PYGZgt{} }\PYG{n}{xv}\PYG{p}{,} \PYG{n}{yv} \PYG{o}{=} \PYG{n}{meshgrid}\PYG{p}{(}\PYG{n}{x}\PYG{p}{,} \PYG{n}{y}\PYG{p}{)}
\PYG{g+gp}{\PYGZgt{}\PYGZgt{}\PYGZgt{} }\PYG{n}{xv}
\PYG{g+go}{array([[ 0. ,  0.5,  1. ],}
\PYG{g+go}{       [ 0. ,  0.5,  1. ]])}
\PYG{g+gp}{\PYGZgt{}\PYGZgt{}\PYGZgt{} }\PYG{n}{yv}
\PYG{g+go}{array([[ 0.,  0.,  0.],}
\PYG{g+go}{       [ 1.,  1.,  1.]])}
\PYG{g+gp}{\PYGZgt{}\PYGZgt{}\PYGZgt{} }\PYG{n}{xv}\PYG{p}{,} \PYG{n}{yv} \PYG{o}{=} \PYG{n}{meshgrid}\PYG{p}{(}\PYG{n}{x}\PYG{p}{,} \PYG{n}{y}\PYG{p}{,} \PYG{n}{sparse}\PYG{o}{=}\PYG{n+nb+bp}{True}\PYG{p}{)}  \PYG{c+c1}{\PYGZsh{} make sparse output arrays}
\PYG{g+gp}{\PYGZgt{}\PYGZgt{}\PYGZgt{} }\PYG{n}{xv}
\PYG{g+go}{array([[ 0. ,  0.5,  1. ]])}
\PYG{g+gp}{\PYGZgt{}\PYGZgt{}\PYGZgt{} }\PYG{n}{yv}
\PYG{g+go}{array([[ 0.],}
\PYG{g+go}{       [ 1.]])}
\end{Verbatim}

\emph{meshgrid} is very useful to evaluate functions on a grid.

\begin{Verbatim}[commandchars=\\\{\}]
\PYG{g+gp}{\PYGZgt{}\PYGZgt{}\PYGZgt{} }\PYG{n}{x} \PYG{o}{=} \PYG{n}{np}\PYG{o}{.}\PYG{n}{arange}\PYG{p}{(}\PYG{o}{\PYGZhy{}}\PYG{l+m+mi}{5}\PYG{p}{,} \PYG{l+m+mi}{5}\PYG{p}{,} \PYG{l+m+mf}{0.1}\PYG{p}{)}
\PYG{g+gp}{\PYGZgt{}\PYGZgt{}\PYGZgt{} }\PYG{n}{y} \PYG{o}{=} \PYG{n}{np}\PYG{o}{.}\PYG{n}{arange}\PYG{p}{(}\PYG{o}{\PYGZhy{}}\PYG{l+m+mi}{5}\PYG{p}{,} \PYG{l+m+mi}{5}\PYG{p}{,} \PYG{l+m+mf}{0.1}\PYG{p}{)}
\PYG{g+gp}{\PYGZgt{}\PYGZgt{}\PYGZgt{} }\PYG{n}{xx}\PYG{p}{,} \PYG{n}{yy} \PYG{o}{=} \PYG{n}{meshgrid}\PYG{p}{(}\PYG{n}{x}\PYG{p}{,} \PYG{n}{y}\PYG{p}{,} \PYG{n}{sparse}\PYG{o}{=}\PYG{n+nb+bp}{True}\PYG{p}{)}
\PYG{g+gp}{\PYGZgt{}\PYGZgt{}\PYGZgt{} }\PYG{n}{z} \PYG{o}{=} \PYG{n}{np}\PYG{o}{.}\PYG{n}{sin}\PYG{p}{(}\PYG{n}{xx}\PYG{o}{*}\PYG{o}{*}\PYG{l+m+mi}{2} \PYG{o}{+} \PYG{n}{yy}\PYG{o}{*}\PYG{o}{*}\PYG{l+m+mi}{2}\PYG{p}{)} \PYG{o}{/} \PYG{p}{(}\PYG{n}{xx}\PYG{o}{*}\PYG{o}{*}\PYG{l+m+mi}{2} \PYG{o}{+} \PYG{n}{yy}\PYG{o}{*}\PYG{o}{*}\PYG{l+m+mi}{2}\PYG{p}{)}
\PYG{g+gp}{\PYGZgt{}\PYGZgt{}\PYGZgt{} }\PYG{n}{h} \PYG{o}{=} \PYG{n}{plt}\PYG{o}{.}\PYG{n}{contourf}\PYG{p}{(}\PYG{n}{x}\PYG{p}{,}\PYG{n}{y}\PYG{p}{,}\PYG{n}{z}\PYG{p}{)}
\end{Verbatim}

\end{fulllineitems}



\section{Documentation}
\label{api:documentation}\label{api::doc}\index{enumerate() (built-in function)}

\begin{fulllineitems}
\phantomsection\label{api:enumerate}\pysiglinewithargsret{\bfcode{enumerate}}{\emph{sequence}\optional{, \emph{start=0}}}{}
Return an iterator that yields tuples of an index and an item of the
\emph{sequence}. (And so on.)

\end{fulllineitems}



\subsection{myclass}
\label{api:myclass}\phantomsection\label{api:module-abc_pack.myclass}\index{abc\_pack.myclass (module)}\index{add() (in module abc\_pack.myclass)}

\begin{fulllineitems}
\phantomsection\label{api:abc_pack.myclass.add}\pysiglinewithargsret{\bfcode{add}}{\emph{x}, \emph{y}}{}~\begin{quote}\begin{description}
\item[{Parameters}] \leavevmode\begin{itemize}
\item {} 
\textbf{\texttt{x}} -- the first value to be added

\item {} 
\textbf{\texttt{y}} -- the second, optional

\end{itemize}

\item[{Returns}] \leavevmode
the sum of the two

\item[{Example}] \leavevmode
add(1.,2.) \# == 3

\end{description}\end{quote}


\strong{See also:}


myclass.public\_service



\end{fulllineitems}

\index{get\_class() (in module abc\_pack.myclass)}

\begin{fulllineitems}
\phantomsection\label{api:abc_pack.myclass.get_class}\pysiglinewithargsret{\bfcode{get\_class}}{\emph{x}}{}
returns a member of myclass right away
\begin{quote}\begin{description}
\item[{Parameters}] \leavevmode
\textbf{\texttt{x}} (\emph{\texttt{array\_like}}) -- means something, but forgot...

\item[{Returns}] \leavevmode
\textbf{u} --
the result

\item[{Return type}] \leavevmode
array

\end{description}\end{quote}

and we have a snippet for you!

\begin{Verbatim}[commandchars=\\\{\}]
\PYG{n}{a} \PYG{o}{=} \PYG{n}{get\PYGZus{}class}\PYG{p}{(}\PYG{p}{)}
\PYG{n}{a}\PYG{o}{.}\PYG{n}{kiss}\PYG{p}{(}\PYG{p}{)}
\end{Verbatim}

\end{fulllineitems}

\index{myclass (class in abc\_pack.myclass)}

\begin{fulllineitems}
\phantomsection\label{api:abc_pack.myclass.myclass}\pysigline{\strong{class }\bfcode{myclass}}
a very fine class indeed!
\index{startme() (myclass method)}

\begin{fulllineitems}
\phantomsection\label{api:abc_pack.myclass.myclass.startme}\pysiglinewithargsret{\bfcode{startme}}{}{}
starts the class and makes it run

\end{fulllineitems}


\end{fulllineitems}

\index{public\_service() (in module abc\_pack.myclass)}

\begin{fulllineitems}
\phantomsection\label{api:abc_pack.myclass.public_service}\pysiglinewithargsret{\bfcode{public\_service}}{}{}
sowas
\begin{quote}\begin{description}
\item[{Parameters}] \leavevmode\begin{itemize}
\item {} 
\textbf{\texttt{x2,..., xn}} (\emph{\texttt{x1,}}) -- 1-D arrays representing the coordinates of a grid.

\item {} 
\textbf{\texttt{indexing}} (\emph{\texttt{\{'xy', 'ij'\}, optional}}) -- 
Cartesian (`xy', default) or matrix (`ij') indexing of output.
See Notes for more details.

\DUspan{versionmodified}{New in version 1.7.0.}


\item {} 
\textbf{\texttt{sparse}} (\emph{\texttt{bool, optional}}) -- If True a sparse grid is returned in order to conserve memory.
Default is False.

\end{itemize}

\end{description}\end{quote}

\end{fulllineitems}



\section{Beam propagation}
\label{bpm:beam-propagation}\label{bpm::doc}
{\hspace*{\fill}\includegraphics{{logo}.png}\hspace*{\fill}}
\index{psf() (in module bpm)}

\begin{fulllineitems}
\phantomsection\label{bpm:bpm.psf}\pysiglinewithargsret{\bfcode{psf}}{\emph{shape}, \emph{units}, \emph{lam}, \emph{NA}, \emph{n0=1.0}, \emph{n\_integration\_steps=200}, \emph{return\_field=False}}{}~\begin{quote}\begin{description}
\item[{Parameters}] \leavevmode\begin{itemize}
\item {} 
\textbf{\texttt{shape}} (\emph{\texttt{Nx,Ny,Nz}}) -- the shape of the geometry

\item {} 
\textbf{\texttt{units}} (\emph{\texttt{dx,dy,dz}}) -- the pixel sizes in microns

\item {} 
\textbf{\texttt{lam}} (\emph{\texttt{float}}) -- the wavelength

\item {} 
\textbf{\texttt{NA}} -- 

\item {} 
\textbf{\texttt{n0}} -- 

\item {} 
\textbf{\texttt{n\_integration\_steps}} -- 

\item {} 
\textbf{\texttt{return\_field}} -- 

\end{itemize}

\item[{Returns}] \leavevmode
\begin{itemize}
\item {} 
\emph{calculates the 3d psf for a perfect, aberration free optical system}

\item {} 
\emph{via the vectorial debye diffraction integral}

\item {} 
\emph{the psf is centered at a grid of given size with voxelsizes units}

\end{itemize}


\end{description}\end{quote}

see \footnote[1]{
Matthew R. Foreman, Peter Toeroek, \emph{Computational methods in vectorial imaging}, Journal of Modern Optics, 2011, 58, 5-6, 339
}

returns:
u, the (not normalized) intensity

or if return\_field = True
u,ex,ey,ez

NA can be either a single number or an even length list of NAs (for bessel beams), e.g.
NA = {[}.1,.2,.5,.6{]} lets light through the annulus .1\textless{}.2 and .5\textless{}.6
\paragraph{References}

\end{fulllineitems}

\index{psf\_u0() (in module bpm)}

\begin{fulllineitems}
\phantomsection\label{bpm:bpm.psf_u0}\pysiglinewithargsret{\bfcode{psf\_u0}}{\emph{shape}, \emph{units}, \emph{zfoc=0}, \emph{NA=0.4}, \emph{lam=0.5}, \emph{n0=1.0}, \emph{n\_integration\_steps=200}}{}
calculates initial plane u0 of a beam focused at zfoc
shape = (Nx,Ny)
units = (dx,dy)
NAs = e.g. (0,.6)

\end{fulllineitems}

\index{bpm\_3d() (in module bpm)}

\begin{fulllineitems}
\phantomsection\label{bpm:bpm.bpm_3d}\pysiglinewithargsret{\bfcode{bpm\_3d}}{\emph{size}, \emph{units}, \emph{lam=0.5}, \emph{u0=None}, \emph{dn=None}, \emph{subsample=1}, \emph{n\_volumes=1}, \emph{n0=1.0}, \emph{return\_scattering=False}, \emph{return\_g=False}, \emph{return\_full=True}, \emph{absorbing\_width=0}, \emph{use\_fresnel\_approx=False}, \emph{scattering\_plane\_ind=0}}{}
simulates the propagation of monochromativ wave of wavelength lam with initial conditions u0 along z in a media filled with dn

size     -    the dimension of the image to be calulcated  in pixels (Nx,Ny,Nz)
units    -    the unit lengths of each dimensions in microns
lam      -    the wavelength
u0       -    the initial field distribution, if u0 = None an incident  plane wave is assumed
dn       -    the refractive index of the medium (can be complex)
n0       -    refractive index of surrounding medium
return\_full - if True, returns the complex field in volume otherwise only last plane

\end{fulllineitems}



\section{Docstring examples}
\label{docstring_examples:docstring-examples}\label{docstring_examples:module-abc_pack.docstring_examples}\label{docstring_examples::doc}\index{abc\_pack.docstring\_examples (module)}
Huhu
\index{citing\_me() (in module abc\_pack.docstring\_examples)}

\begin{fulllineitems}
\phantomsection\label{docstring_examples:abc_pack.docstring_examples.citing_me}\pysiglinewithargsret{\bfcode{citing\_me}}{}{}
please cite \footnote[1]{
Hugo Beierthal (2013) \emph{fastsomething: We wish you a merry christmas}
}

and we do have footnotes \footnote[2]{
Text fo the footnote
} as well!
\begin{gather}
\begin{split}e^{-\alpha x} = \int_0^1 dk f(k)\end{split}\notag
\end{gather}\paragraph{References}

\end{fulllineitems}

\index{google\_style() (in module abc\_pack.docstring\_examples)}

\begin{fulllineitems}
\phantomsection\label{docstring_examples:abc_pack.docstring_examples.google_style}\pysiglinewithargsret{\bfcode{google\_style}}{\emph{x}, \emph{y=None}, \emph{fname='`}}{}~\begin{quote}\begin{description}
\item[{Parameters}] \leavevmode\begin{itemize}
\item {} 
\textbf{\texttt{x}} (\emph{\texttt{int}}) -- the first value

\item {} 
\textbf{\texttt{y}} (\emph{\texttt{float}}) -- nothing

\item {} 
\textbf{\texttt{fname}} (\emph{\texttt{str}}) -- obvious

\end{itemize}

\item[{Returns}] \leavevmode
everything you never cared about

\end{description}\end{quote}

\end{fulllineitems}

\index{numpy\_style() (in module abc\_pack.docstring\_examples)}

\begin{fulllineitems}
\phantomsection\label{docstring_examples:abc_pack.docstring_examples.numpy_style}\pysiglinewithargsret{\bfcode{numpy\_style}}{}{}
Return a new array of given shape and type, filled with zeros.
\begin{quote}\begin{description}
\item[{Parameters}] \leavevmode\begin{itemize}
\item {} 
\textbf{\texttt{shape}} (\emph{\texttt{int or sequence of ints}}) -- Shape of the new array, e.g., \code{(2, 3)} or \code{2}.

\item {} 
\textbf{\texttt{dtype}} (\emph{\texttt{data-type, optional}}) -- The desired data-type for the array, e.g., \emph{numpy.int8}.  Default is
\emph{numpy.float64}.

\item {} 
\textbf{\texttt{order}} (\emph{\texttt{\{'C', 'F'\}, optional}}) -- Whether to store multidimensional data in C- or Fortran-contiguous
(row- or column-wise) order in memory.

\end{itemize}

\item[{Returns}] \leavevmode
\textbf{out} --
Array of zeros with the given shape, dtype, and order.

\item[{Return type}] \leavevmode
ndarray

\end{description}\end{quote}


\strong{See also:}

\begin{description}
\item[{\code{zeros\_like()}}] \leavevmode
Return an array of zeros with shape and type of input.

\item[{\code{ones\_like()}}] \leavevmode
Return an array of ones with shape and type of input.

\item[{\code{empty\_like()}}] \leavevmode
Return an empty array with shape and type of input.

\item[{\code{ones()}}] \leavevmode
Return a new array setting values to one.

\item[{\code{empty()}}] \leavevmode
Return a new uninitialized array.

\end{description}


\paragraph{Examples}

\begin{Verbatim}[commandchars=\\\{\}]
\PYG{g+gp}{\PYGZgt{}\PYGZgt{}\PYGZgt{} }\PYG{n}{np}\PYG{o}{.}\PYG{n}{zeros}\PYG{p}{(}\PYG{l+m+mi}{5}\PYG{p}{)}
\PYG{g+go}{array([ 0.,  0.,  0.,  0.,  0.])}
\end{Verbatim}

\begin{Verbatim}[commandchars=\\\{\}]
\PYG{g+gp}{\PYGZgt{}\PYGZgt{}\PYGZgt{} }\PYG{n}{np}\PYG{o}{.}\PYG{n}{zeros}\PYG{p}{(}\PYG{p}{(}\PYG{l+m+mi}{5}\PYG{p}{,}\PYG{p}{)}\PYG{p}{,} \PYG{n}{dtype}\PYG{o}{=}\PYG{n}{np}\PYG{o}{.}\PYG{n}{int}\PYG{p}{)}
\PYG{g+go}{array([0, 0, 0, 0, 0])}
\end{Verbatim}

\begin{Verbatim}[commandchars=\\\{\}]
\PYG{g+gp}{\PYGZgt{}\PYGZgt{}\PYGZgt{} }\PYG{n}{np}\PYG{o}{.}\PYG{n}{zeros}\PYG{p}{(}\PYG{p}{(}\PYG{l+m+mi}{2}\PYG{p}{,} \PYG{l+m+mi}{1}\PYG{p}{)}\PYG{p}{)}
\PYG{g+go}{array([[ 0.],}
\PYG{g+go}{       [ 0.]])}
\end{Verbatim}

\begin{Verbatim}[commandchars=\\\{\}]
\PYG{g+gp}{\PYGZgt{}\PYGZgt{}\PYGZgt{} }\PYG{n}{s} \PYG{o}{=} \PYG{p}{(}\PYG{l+m+mi}{2}\PYG{p}{,}\PYG{l+m+mi}{2}\PYG{p}{)}
\PYG{g+gp}{\PYGZgt{}\PYGZgt{}\PYGZgt{} }\PYG{n}{np}\PYG{o}{.}\PYG{n}{zeros}\PYG{p}{(}\PYG{n}{s}\PYG{p}{)}
\PYG{g+go}{array([[ 0.,  0.],}
\PYG{g+go}{       [ 0.,  0.]])}
\end{Verbatim}

\begin{Verbatim}[commandchars=\\\{\}]
\PYG{g+gp}{\PYGZgt{}\PYGZgt{}\PYGZgt{} }\PYG{n}{np}\PYG{o}{.}\PYG{n}{zeros}\PYG{p}{(}\PYG{p}{(}\PYG{l+m+mi}{2}\PYG{p}{,}\PYG{p}{)}\PYG{p}{,} \PYG{n}{dtype}\PYG{o}{=}\PYG{p}{[}\PYG{p}{(}\PYG{l+s+s1}{\PYGZsq{}}\PYG{l+s+s1}{x}\PYG{l+s+s1}{\PYGZsq{}}\PYG{p}{,} \PYG{l+s+s1}{\PYGZsq{}}\PYG{l+s+s1}{i4}\PYG{l+s+s1}{\PYGZsq{}}\PYG{p}{)}\PYG{p}{,} \PYG{p}{(}\PYG{l+s+s1}{\PYGZsq{}}\PYG{l+s+s1}{y}\PYG{l+s+s1}{\PYGZsq{}}\PYG{p}{,} \PYG{l+s+s1}{\PYGZsq{}}\PYG{l+s+s1}{i4}\PYG{l+s+s1}{\PYGZsq{}}\PYG{p}{)}\PYG{p}{]}\PYG{p}{)} \PYG{c+c1}{\PYGZsh{} custom dtype}
\PYG{g+go}{array([(0, 0), (0, 0)],}
\PYG{g+go}{      dtype=[(\PYGZsq{}x\PYGZsq{}, \PYGZsq{}\PYGZlt{}i4\PYGZsq{}), (\PYGZsq{}y\PYGZsq{}, \PYGZsq{}\PYGZlt{}i4\PYGZsq{})])}
\end{Verbatim}

\end{fulllineitems}

\index{rst\_style() (in module abc\_pack.docstring\_examples)}

\begin{fulllineitems}
\phantomsection\label{docstring_examples:abc_pack.docstring_examples.rst_style}\pysiglinewithargsret{\bfcode{rst\_style}}{\emph{x}, \emph{y=None}, \emph{fname='`}}{}~\begin{quote}\begin{description}
\item[{Parameters}] \leavevmode\begin{itemize}
\item {} 
\textbf{\texttt{path}} (\emph{\texttt{str}}) -- The path of the file to wrap

\item {} 
\textbf{\texttt{field\_storage}} (\emph{\texttt{FileStorage}}) -- The \code{FileStorage} instance to wrap

\item {} 
\textbf{\texttt{temporary}} -- Whether or not to delete the file when the File

\end{itemize}

\item[{Returns}] \leavevmode
A buffered writable file descriptor

\item[{Return type}] \leavevmode
BufferedFileStorage

\end{description}\end{quote}

\end{fulllineitems}

\index{sphynx\_style() (in module abc\_pack.docstring\_examples)}

\begin{fulllineitems}
\phantomsection\label{docstring_examples:abc_pack.docstring_examples.sphynx_style}\pysiglinewithargsret{\bfcode{sphynx\_style}}{\emph{x}, \emph{y=None}, \emph{fname='`}}{}~\begin{quote}\begin{description}
\item[{Parameters}] \leavevmode\begin{itemize}
\item {} 
\textbf{\texttt{x}} (\emph{\texttt{str}}) -- the first value

\item {} 
\textbf{\texttt{y}} (\emph{\texttt{float, int, ndarray}}) -- the second value

\item {} 
\textbf{\texttt{fname}} (\emph{\texttt{str}}) -- the name tow rite to

\end{itemize}

\item[{Returns}] \leavevmode
1 on sucess

\end{description}\end{quote}

\end{fulllineitems}

\phantomsection\label{docstring_examples:module-abc_pack}\index{abc\_pack (module)}
Every great project starts with a line
\begin{itemize}
\item {} 
\DUspan{xref,std,std-ref}{genindex}

\item {} 
\DUspan{xref,std,std-ref}{modindex}

\item {} 
\DUspan{xref,std,std-ref}{search}

\end{itemize}


\renewcommand{\indexname}{Python Module Index}
\begin{theindex}
\def\bigletter#1{{\Large\sffamily#1}\nopagebreak\vspace{1mm}}
\bigletter{a}
\item {\texttt{abc\_pack}}, \pageref{docstring_examples:module-abc_pack}
\item {\texttt{abc\_pack.docstring\_examples}}, \pageref{docstring_examples:module-abc_pack.docstring_examples}
\end{theindex}

\renewcommand{\indexname}{Index}
\printindex
\end{document}
